%---------------------------------------------------
% This CV was inspired in Dario Taraborelli template
%---------------------------------------------------

%!TEX TS-program = xelatex
%!TEX encoding = UTF-8 Unicode

\documentclass[9pt, a4paper, pstricks]{article}
\usepackage{fontspec}

% DOCUMENT LAYOUT
\usepackage{geometry}
\usepackage{changepage}
\geometry{a4paper, textwidth=5.5in, textheight=9.2in, marginparsep=7pt, marginparwidth=.6in,left=4cm}
\setlength\parindent{0in}

% FONTS
\usepackage{xunicode}
\usepackage{xltxtra}
\defaultfontfeatures{Mapping=tex-text}
\setromanfont [ItalicFeatures={Ligatures={Common,Rare}}, Numbers={OldStyle}, BoldFont=Hoefler Text Bold SC]{Minion Pro}
\setmonofont[Scale=0.8]{Monaco}

% ---- CUSTOM SYMBOLS
\newcommand{\amper}{{\fontspec[Scale=.95]{Hoefler Text}\selectfont\itshape\&}}
\newcommand{\current}{{\hspace{-0.97em}\color{feup}$\star$}~}
\newcommand{\since}{{\color{feup}---}}

% ---- MARGIN YEARS
\usepackage{marginnote}
\newcommand{\years}[1]{\marginnote{\small #1}}
\newcommand{\bigyears}[1]{\marginnote{#1}}
\newcommand{\marginimage}{
%\marginnote{
%\vspace{-1cm}~\\
%{\hspace{-2.2cm}\includegraphics[scale=0.5]{photo.jpg}}
%}
}
\renewcommand*{\raggedleftmarginnote}{}
\setlength{\marginparsep}{7pt}
\reversemarginpar

\newcommand{\tp}{{\tikz\draw[black,fill=black] (0,0) circle (.5ex);}}
\newcommand{\htp}{{\tikz\draw[feup,fill=feup] (0,0) circle (.5ex);}}
\newcommand{\ttp}{{\tikz\draw[black,fill=black] (0,0) circle (.5ex);}}
\newcommand{\tpp}{{\tikz\draw[black,fill=black] (0,0) circle (.5ex);}}
\newcommand{\nth}{\textsuperscript{th}~}
\newcommand{\nd}{\textsuperscript{nd}~}
\newcommand{\st}{\textsuperscript{st}~}
\newcommand{\rd}{\textsuperscript{rd}~}
\newcommand{\inurl}[1]{~\{~{\small\url{#1}~}\}}
\newcommand{\acr}[1]{{\sc #1}}
\newcommand{\rnd}{{\sc r\amper d}~}
\newcommand{\smallplus}{\raise.50ex\hbox{\tiny{+}}}
\newcommand{\smallscplus}{\raise.30ex\hbox{\tiny{+}}}
\newcommand{\cpp}{c\smallscplus\smallscplus}
\newcommand{\reg}{\textsuperscript{\textregistered}}

\newcommand{\publication}[4]{\years{#1}#2,~\emph{``#3''}, #4}
\newcommand{\awardedpub}[4]{\years{\current~#1}#2,~\emph{``#3''}, #4 {\color{feup} \emph{Best Paper Award.}}}

%
\usepackage[final=true]{microtype}
\linespread{1.1}

% No orphans
\clubpenalty = 500
\widowpenalty = 1000

% HEADINGS
\usepackage{sectsty}
\usepackage[normalem]{ulem}
\sectionfont{\color{feup}\rmfamily\mdseries\large}
\subsectionfont{\rmfamily\mdseries\scshape\normalsize}
\subsubsectionfont{\rmfamily\bfseries\upshape\normalsize}

% PDF SETUP
% ---- FILL IN HERE THE DOC TITLE AND AUTHOR
\usepackage[xetex, bookmarks, colorlinks, breaklinks, pdftitle={Curriculum Vitae of Hugo Sereno Ferreira},pdfauthor={Hugo Sereno Ferreira}]{hyperref}

% COLORS
\usepackage[table]{xcolor}
%\definecolor{feup}{cmyk}{0.2,0.9,1.0,0.2}
\definecolor{feup}{rgb}{0.7,0.0,0.0}

% INVERTED COLORS
% \definecolor{feup}{cmyk}{0,0,0,0}
% \definecolor{coolblack}{rgb}{0.14,0.12,0.126}
% \definecolor{coolwhite}{cmyk}{0,0,0,0.2}
% \usepackage{background}
% \SetBgContents{}

% HYPERREF
\hypersetup{linkcolor=feup,citecolor=feup,filecolor=black,urlcolor=feup}

% HYPHENATION
\hyphenation{Chi-ca-go}
\hyphenation{a-da-pta-ble}
\hyphenation{Re-Think-ing}

\usepackage{tikz}
\usepackage{booktabs}

% SPARKLINES

\RequirePackage{sparklines}

\colorlet{sparkspikecolor}{feup}
\setlength{\sparkspikewidth}{5pt}

\newcommand{\ratiofivespark}[8]{
	\begin{sparkline}{6}
	  \sparkspike  .05 \ratio{#2}{#1}
	  \sparkspike  .3 \ratio{#3}{#1}
	  \sparkspike  .55 \ratio{#4}{#1}
	  \sparkspike  .80 \ratio{#5}{#1}
	  \sparkspike 1.05 \ratio{#6}{#1}
    \sparkspike 1.30 \ratio{#7}{#1}
    \sparkspike 1.55 \ratio{#8}{#1}
	\end{sparkline}
}

% DOCUMENT
\begin{document}

% INVERTED COLORS
% \pagecolor{coolblack}
% \color{coolwhite}

{\huge\sc\color{feup}{\addfontfeature{LetterSpace=3.0} hugo sereno ferreira, phd}}\\
{\small {\scshape  {\bfseries birth}\hspace{2mm}7/1980 \hspace{8mm} {\bfseries e}\hspace{2mm}hugo.sereno@fe.up.pt \hspace{8mm} {\bfseries t}\hspace{2mm}220301567}}\\

\years{\vspace{-4mm}\\\hspace*{-10mm}\includegraphics[scale=1.2]{qrcode.eps}}\marginimage{\small Professor (7yrs) at {\sc feup}, researcher (8yrs) and entrepreneur (4yrs) with 14+ years of professional experience, 25+ published works (\emph{h-index 7}, \emph{i10-index 6}), and 20+ MSc supervisions, focused in high-quality, adaptable, large-scale, \emph{big-data} software solutions. His career encompasses projects for the portuguese and international military, local government and privately held businesses. Nowadays he focus on Object-Functional Programming, Architecture \amper~Design Patterns, and Machine Learning. First student to be jointly awarded a Ph.D. (with \emph{distinction}) in \emph{Computer Science} by the Universities of Porto {\sc (up)}, Minho {\sc (um)} and Aveiro {\sc (ua)}, he later held a \emph{postdoc} research position at \acr{inesc tec} (1.5yrs) on \emph{Large Scale Software Systems} for \emph{Ambient Assisted Living}.

\vspace{1mm}After receiving a \emph{LicEng} degree in \emph{Informatics and Computation Engineering}, he performed military \rnd for \acr{nato} at \emph{ParadigmaXis, S.A.} There he progressively assumed the roles of developer, team leader, software architect, and project manager in several government and privately held projects. His university teaching career started shortly after becoming a doctorate candidate in \emph{meta-programming} and \emph{model-driven software engineering} (2007), mainly at the Faculty of Engineering of the University of Porto (\acr{feup}).

\vspace{1mm}Early 2011 he co-founded \emph{Shiftforward}, an international strategy and technology consultancy company specialised in \emph{big-data} software solutions for the \emph{Online Advertising} industry. In 2014, after raising 1.2{\sc m}\$ in a seed {\sc vc} funding round from Portugal Ventures and industry veterans and private investors, Brian Fitzpatrick and Florian Heinemann, Shiftforward's business model evolved into a \emph{framework-as-a-service}, lead by its two flag products, the \emph{AdStax} and the \emph{AdForecaster}, a state-of-the-art forecasting engine, able to predict the performance of multiple campaigns with arbitrary segments and targeting rules, in systems with billions ($10^{12}$) of impressions, and simulate alternative \emph{``what-if''} scenarios in few seconds.}

%%\hrule
\section*{Professional experience}

\subsection*{currently held positions \amper~affiliations}
\noindent
\years{\current~2013 ---}\emph{Assistant Professor}, {\sc feup} Faculty of Engineering, University of Porto~\inurl{http://www.fe.up.pt} \\
\years{\current~2014 ---}\emph{Member of the Board of Directors \amper~Shareholder}, Shiftforward, S.A.~\inurl{http://shiftforward.eu}\\
\years{\current~2011 ---}\emph{Co-Founder \amper~Chief Technology Officer} ({\sc cto}), Shiftforward, S.A.~\inurl{http://shiftforward.eu}\\
\years{\current~2009 ---}\emph{Senior Researcher}, {\sc inesc} Instituto de Engenharia de Sistemas e Computadores do Porto

\subsection*{past experience in academy}
\noindent
\years{2008 --- 13}\emph{Assistant Lecturer}, {\sc feup} Faculty of Engineering, University of Porto~\inurl{http://www.fe.up.pt} \\
\years{2011 --- 13}\emph{Collaborating Lecturer}, {\sc flup} Faculty of Arts, University of Porto~\inurl{http://www.letras.up.pt} \\
\years{2011 --- 12}\emph{Postdoctoral Researcher}, {\sc inesc} Instituto de Engenharia de Sistemas e Computadores do Porto\\
\years{2011}\emph{Guest Lecturer}, {\sc isep} Instituto Superior de Engenharia do Porto~\inurl{http://www.isep.ipp.pt}\\
\years{2008}\emph{Lecturer}, {\sc istec} Instituto Superior de Tecnologias Avançadas~\inurl{http://www.istec.pt}

\subsection*{past experience in industry}
\noindent
\years{2011}\emph{Senior Architect}, ParadigmaXis - Arquitectura e Engenharia de Software, S.A.\\
\years{2010 --- 11}\emph{Independent Consultant} working with Critical Software {\sc pt}, Semasio {\sc de}, SAGE {\sc pt}, Tectonic {\sc uk}, etc.\\
\years{2007 --- 10}\emph{Researcher}, ParadigmaXis - Arquitectura e Engenharia de Software, S.A.\\
\years{2003 --- 07}\emph{Software Engineer}, ParadigmaXis - Arquitectura e Engenharia de Software, S.A.\\
\years{2001 --- 03}\emph{Freelancer} in developing software solutions for medical healthcare and sports centers

\section*{Education}

\noindent\years{2007 --- 11}\textsc{Ph.D.} in Computer Science (w/ distinction), {\sc mapi} Joint Doctoral Programme, {\sc feup\amper fcup} {\sc um} {\sc ua}\\
\years{2006}\textsc{LicEng} in Informatics and Computation Engineering, {\sc feup}

\section*{Grants, certifications, honors \amper~awards}
\noindent
\years{2012}Certification of \emph{Functional Programming Principles in Scala} with \emph{distinction}, by École Polytechnique Fédérale de Lausanne @ Coursera. \\
\years{2012}Best paper award for Tiago Boldt Sousa, Hugo Sereno Ferreira, \emph{``Object-Functional Patterns: Re-Thinking Development in a Post-Functional World''}, {\sc sedes @ quatic 2012}\\
\years{2011 --- 12}Postdoctoral research grant by {\sc inesc}, \emph{cf.} {\sc bpd/110054/caalyx-mv\_usig}\\
\years{2007 --- 11}Scholarship grant for doctoral studies by {\sc fct}, \emph{cf.} {\sc sfrh/bde/33298/2008} \\
\years{2009}Best paper award for Hugo Sereno Ferreira, Ademar Aguiar, João Pascoal Faria, \emph{``Adaptive Object Modelling: Patterns, Tools and Applications''}, {\sc sedes @ icsea 2009}\\
\years{2008}Best paper award for Hugo Sereno Ferreira, Filipe Correia, Leon Welicki,~\emph{``Patterns for Data and Metadata Evolution in Adaptive Object Models''}, {\sc plop @ oopsla 2008}\\
\years{2004 --- 11}{\sc nato}-secret security clearance for military \rnd

% \section*{Publications\hfill{\href{http://scholar.google.pt/citations?user=bW-JHDEAAAAJ&hl=en}{\small\sc h-index 7 · i10-index 6 · citations 126~~~\ratiofivespark{45}{6}{15}{14}{23}{20}{34}{11}\hspace{0.45cm}}}}
\section*{Publications\hfill{\href{http://scholar.google.pt/citations?user=bW-JHDEAAAAJ&hl=en}{\small\sc h-index 7 · i10-index 6 · citations 126}}}

\subsection*{journal articles~(peer-reviewed)}

\noindent\years{2010}Hugo Sereno Ferreira, Ademar Aguiar, João Pascoal Faria,~\emph{``Adaptive Object-Models: A Research Roadmap''}, International Journal On Advances in Software, Vol 3.

\subsection*{conference articles~(peer-reviewed)}

\noindent
\publication{2015}{Tiago Boldt Sousa, Filipe Correira, Hugo Sereno Ferreira}{DevOps patterns for software orchestration on public and private clouds}{Proceedings of the 22\nd Conference on Pattern Languages of Programs, Pittsburgh, Pennsylvania, USA.}\\
\publication{2014}{Luís Fonseca, Tiago Boldt Sousa, Hugo Sereno Ferreira}{Rapid Prototyping and Development Framework for Android with Scala}{Proceedings of the 6\nth Symposium on Informatics. Porto, Portugal.}\\
\publication{2014}{Omar Castro, Hugo Sereno Ferreira, Tiago Boldt Sousa}{Collaborative Web Platform for UNIX-based Big Data Processing}{Proceedings of the 11\nth International Conference on Cooperative Design, Visualization and Engineering. Seattle, USA. Lecture Notes in Computer Science 8683.}\\
\publication{2014}{Nuno Flores, Ademar Aguiar, Hugo Sereno Ferreira}{The Concept of Ba Applied to Software Knowledge}{Proceedings of the 7\nth International Workshop on Cooperative and Human Aspects of Software Engineering.}\\
\publication{2013}{Bruno Ferreira, Vânia Guimarães, Hugo Sereno Ferreira}{Smartphone Based Fall Prevention Exercises}{Proceedings of the 15\nth {\sc ieee} International Conferences on e-Health Networking, Application~\amper~Services {\sc (healthcom 2013)}. Lisbon, Portugal.}\\
\publication{2013}{Angelo Martins, Tiago Boldt Sousa, João Pascoal Faria, Hugo Sereno Ferreira}{Architecture for an {\sc aal} Ecosystem}{Proceedings of the {6\nth International C* Conference on Computer Science~\amper~Software Engineering}. Porto, Portugal.}\\
\awardedpub{2012}{Tiago Boldt Sousa, Hugo Sereno Ferreira}{Object-Functional Patterns: Re-Thinking Development in a Post-Functional World}{Proceedings of {\sc sedes'2012} at the 8\nth International Conference on the Quality of Information and Communications Technology. Lisbon, Portugal.}\\
\publication{2012}{Sandra Prescher, Alan Bourke, Friedrich Koehler, Angelo Manuel Martins, Hugo Sereno Ferreira, Tiago Boldt Sousa, Rui Castro, António Santos, Marc Torrent Poch, Sergi Gomis Gascó, Hospedales Margarita, John Nelson}{A Ubiquitous Ambient Assisted Living Solution to Promote Safer Independent Living in Older Adults Suffering from Co-morbidity}{Proceedings of the 34\nth Annual International Conference of the {\sc ieee} Engineering in Medicine and Biology Society ({\sc embc}'12). California, USA.}\\
\publication{2012}{Hugo Sereno Ferreira, Tiago Sousa, Angelo Martins}{Scalable Integration of Multiple Health Sensor Data for Observing Medical Patterns}{Proceedings of the 9\nth International Conference on Cooperative Design, Visualization and Engineering. Osaka, Japan. Lecture Notes in Computer Science 7467.}\\
\publication{2012}{Tiago Almeida, Hugo Sereno Ferreira, Tiago Sousa}{A collaborative expandable framework for end-users and programmers}{Proceedings of the 9\nth International Conference on Cooperative Design, Visualization and Engineering. Osaka, Japan. Lecture Notes in Computer Science 7467.}\\
\publication{2011}{Patricia Matsumoto, Eduardo Guerra, Hugo Sereno Ferreira, Filipe Correia, Joseph Yoder, Ademar Aguiar}{AOM Roles Metadata Mapping}{Proceedings of the 18\nth Conference on Pattern Languages of Programs, Portland, Oregon, USA.}\\
\publication{2011}{Hugo Sereno Ferreira, Filipe Correia, Ademar Aguiar, Joseph Yoder}{The Lazy Semantics Pattern on the context of Meta-Architectures}{2\nd Asian Conf. on Pattern Languages of Programs. Tokyo, Japan.}\\
\publication{2010}{Hugo Sereno Ferreira, Filipe Correia, Joseph Yoder, Ademar Aguiar}{Core Patterns of Object-Oriented Meta-Architectures}{Proceedings of the 17\nth Conference on Pattern Languages of Programs, Reno, Nevada, USA.}\\
\publication{2010}{Marco Cunha, Ana Paiva, Hugo Sereno Ferreira, Rui Abreu}{PETTool: A Pattern-Based GUI Testing Tool}{2\nd International Conference on Software Technology and Engineering, Puerto Rico, USA.}\\
\publication{2009}{Gabriela Soares, Rosaldo Rossetti, Nuno Flores, Ademar Aguiar, Hugo Sereno Ferreira}{A Cooperative Personal Agenda in a Collaborative Team Environment}{Proceedings of the 6\nth International Conference on Cooperative Design, Visualization and Engineering. Lecture Notes in Computer Science 5738.}\\
\publication{2009}{António Rito Silva, David Martinho, Ademar Aguiar, Nuno Flores, Filipe Correia, Hugo Sereno Ferreira}{An Implementation Model for Agile Business Process Tools}{International Workshop on Organizational Design and Engineering, Portugal.}\\
\publication{2009}{Hugo Sereno Ferreira, Filipe Correia, Ademar Aguiar}{Design for an Adaptive Object-Model Framework: An Overview}{Proceedings of the 4\nth International Workshop on Models@runtime, Denver, Colorado, USA.}\\
\publication{2009}{Filipe Correia, Hugo Sereno Ferreira, Nuno Flores, Ademar Aguiar}{Patterns for Consistent Software Documentation}{Proceedings of the 16\nth Conference on Pattern Languages of Programs, Chicago, Illinois, USA.}\\
\awardedpub{2009}{Hugo Sereno Ferreira, Ademar Aguiar, João Pascoal Faria}{Adaptive Object Modelling: Patterns, Tools and Applications}{3\rd Symposium on Doctoral Students of Software Engineering. Proceedings of the 4\nth International Conference on Software Engineering Advances, Porto, Portugal.}\\
\publication{2009}{Filipe Correia, Hugo Sereno Ferreira, Nuno Flores, Ademar Aguiar}{Incremental Knowledge Acquisition in Software Development Using a Weakly-Typed Wiki}{Proceedings of the 5\nth International Symposium on Wikis and Open Collaboration, Orlando, Florida, USA.}\\
\awardedpub{2008}{Hugo Sereno Ferreira, Filipe Correia, Leon Welicki}{Patterns for Data and Metadata Evolution in Adaptive Object Models}{Proceedings of 15\nth Conference on Pattern Languages of Programs, Nashville, Tennessee, USA.}\\
\publication{2008}{Filipe Correia and Hugo Sereno Ferreira}{Trends on Adaptive Object-Model Research}{Proceedings of the 3\rd Edition of the Doctoral Symposium in Informatics Engineering, Porto, Portugal.}

\subsection*{technical whitepapers}
\noindent\years{2012}Paulo Cunha, André Silva, João Azevedo, Hugo Sereno Ferreira,~\emph{``AdStress: The Online Advertising Stress Testing Platform''}~\inurl{http://goo.gl/XL20R}.\\
\years{2012}Paulo Cunha, João Azevedo, André Silva, Hugo Sereno Ferreira,~\emph{``AdForecaster: Predict Your Campaign Ad Impressions in Seconds''}~\inurl{http://goo.gl/dPwr0}.

\subsection*{others}
\noindent
\years{2012}Filipe Correia, Nuno Flores, Hugo Sereno Ferreira, Ademar Aguiar,~\emph{``Assessing Tools for Software Development: An overview of three user evaluations''}, User evaluation for Software Engineering Researchers ({\sc user}) Workshop at the 34\nth International Conference on Software Engineering.\\
\years{2012}Hugo Sereno Ferreira,~\emph{``Introdução à Programação Objecto-Funcional com Scala''}, MundoJ 53.\\
\years{2011}Michel Wermelinger, Hugo Sereno Ferreira,~\emph{``Quality evolution track at {\sc quatic}’10''}, {\sc acm sigsoft} Software Engineering Notes, p. 28.

\subsection*{dissertations}

\noindent\years{2010}Adaptive Object-Modeling: Patterns, Tools and Applications. \emph{PhD dissertation}\\
\noindent\years{2004}Framework for Development of Command \amper~Control Systems. \emph{LicEng dissertation}

\section*{Invited talks, Shows \amper~Interviews}

\noindent
\years{2015}\emph{Uber e Negócios Online}, Jornal da Noite @ RTP2\\
\years{2015}\emph{The Peter Fatality: Why People Rise to the Level of their Incompetence}, Panel @ Commit Porto, Faculty of Engineering, University of Porto\\
\years{2015}\emph{ShiftForward}, JobIT 2015, Faculty of Engineering, University of Porto\\
\years{2015}\emph{Software and Services}, Panel, 8\nth Congress on Services and Management Engineering ({\sc cesg 2014})\\
\years{2014}\emph{ShiftForward}, JobIT 2014, Faculty of Engineering, University of Porto\\
\years{2014}\emph{Computer Science Challenges from the Industry}, Panel, INForum 2014, Porto\\
\years{2014}\emph{Why Post-Functional Programming Matters}, Ripple Conference 2014, Porto\\
\years{2013}\emph{Workshop: Introduction to Scala}, Faculty of Engineering, University of Porto\\
\years{2013}\emph{Interview with ShiftForward}, {\sc tsf} Mundo Novo (radio program on entrepreneurship)\\
\years{2013}\emph{Why Post-Functional Programming Matters}, 8\nth National Meeting of Informatics Students ({\sc enei 2013})\\
\years{2013}\emph{Shiftforward -- Lessons Learned in Entrepreneurship}, Beta-Talk Porto, Facts Coworking\\
\years{2012}Episode \emph{``What's a Programming Language?''} in \emph{One Minute Engineering}~\inurl{paginas.fe.up.pt/~engmin}\\
\years{2011}\emph{Incomplete by Design}, Inside Awareness, Faculty of Engineering, University of Porto\\
\years{2009}\emph{Causal Connections}, Models@Runtime Workshop, Denver, Colorado, USA\\
\years{2009}\emph{Pattern Languages}, Seminários Ortogonais, Faculty of Sciences, University of Porto\\
\years{2008}\emph{The Path to Abstraction}, Instituto Superior de Tecnologias Avançadas

\section*{Educational services}

\subsection*{phd in informatics engineering supervising · prodei · feup}
\noindent\years{\current~2011 ---}Tiago Boldt Sousa, \emph{Object-Functional Patterns: Rethinking Development in a Postfunctional World}.

\subsection*{master degree in informatics engineering supervising · feup}
\noindent
\years{\current~2015 --- 16}Gabriel Candal, \emph{Exploring Visual Programming Concepts for Probabilistic Programming Languages} at {\sc feup}. \\
\years{\current~2015 --- 16}Pedro Fernandes, \emph{Framework for Monte Carlo Tree Search-related strategies in Competitive Card Based Games} at {\sc feup}. \\
\years{\current~2015 --- 16}Duarte Duarte, \emph{Framework for Multi-Agent Simulation of User Behavior in E-Commerce Sites} at {\sc shiftforward}. \\
\years{\current~2015 --- 16}João Pedro Dias, \emph{Reverse Engineering Static Content and Dynamic Behaviour of E-Commerce Sites for Fun and Profit} at {\sc shiftforward}. \\
\years{2014 --- 15}Filipe Oliveira, \emph{Exploring Macros for Model-Based Generation of Type-safe REST services in Scala} at {\sc feup}. \\
\years{2013 --- 14}Luís Fonseca, \emph{Exploring Rapid Application Development for Android with Scala and SBT} at {\sc inesc tec}.\\
\years{2013 --- 14}Bruno Maia, \emph{Transformation Patterns for a Reactive application} at {\sc blip}.\\
\years{2013 --- 14}Omar Castro, \emph{Shellhive: Towards a Collaborative Visual Programming Language for UNIX Workflows} at {\sc feup}.\\
\years{2013 --- 14}Jorge Silva, \emph{The Road to Enlightenment: Generating Insight and Predicting Consumer Actions in Digital Markets} at {\sc shiftforward}.\\
\years{2013 --- 14}João Quarteu, \emph{Towards a Self-Managed Framework for Orchestration and Integration of Devices in AAL} at {\sc inesc tec}.\\
%\years{\current~2012 ---}Luis Simões, \emph{Pattern Language for Managing Open Source Projects}.\\
\years{2012 --- 13}João Figueiredo, \emph{Modularization of Large Web Applications} at {\sc blip}.\\
\years{2012 --- 13}Carlos Babo, \emph{Generic and parameterizable service for remote configuration of mobile phones using Near Field Communication} at Fraunhofer {\sc aicos}.\\
\years{2012 --- 13}Bruno Ferreira, \emph{Smartphone Based Fall Prevention Exercises} at Fraunhofer {\sc aicos}.\\
\years{2012 --- 13}Vasco Grilo, \emph{Towards a Live Development Environment} at {\sc feup}.\\
\years{2011 --- 12}Tiago Almeida, \emph{End-User Programming In Mobile Devices through Reusuable Visual Components Composition} at {\sc feup}.\\
\years{2011 --- 12}Jorge Mateus, \emph{Agile Business Intelligence Using Microsoft\reg~SharePoint} at {\sc critical manufacturing}.\\
\years{2011 --- 12}Inês Carvalho, \emph{Espresso Medical Systems} at {\sc scionis}.\\
\years{2010 --- 11}André Carmo, \emph{Introducing End-User Reconfiguration on Clinical Knowledge Information Systems} at {\sc critical software, s.a.}\\
\years{2010 --- 11}Marcelo Cerqueira, \emph{Ambiente de Modelação e Configuração de Processos} at {\sc siemens s.a.}

\subsection*{master degree in electrical and computer engineering supervising · feup}
\noindent
\years{2014 --- 15}João Silva, \emph{Bluetooth based Warning System for Ambient Assisted Living} at Fraunhofer {\sc aicos}.

\subsection*{master degree in informatics engineering co-supervising · feup}
\noindent
\years{\current~2015 ---}Miguel Oliveira, \emph{Ranking Personal Information} at {\sc mailcube}. Supervised by \emph{Prof. Ademar Aguiar}.\\
\years{2013 --- 14}Pedro Borges, \emph{Online Advertising: Forecasting and Synthesising Web Activity Based On Historical Data} at {\sc shiftforward}.  Supervised by \emph{Prof. João Mendes Moreira}.\\
\years{2009 --- 10}João Gradim Pereira, \emph{Improving Variability of Applications using Adaptive Object-Models} at Tecla Colorida, Lda. Supervised by \emph{Prof. Ademar Aguiar}.

\subsection*{bachelor degree in informatics engineering co-supervising · isep}
\noindent
\years{2012}Diogo Silva, \emph{Desenvolvimento de Aplicação Android para Monitorização de Idosos} at {\sc inesc}. Supervised by \emph{Prof. Angelo Martins}.\\
\years{2012}Vítor Moreira, \emph{Desenvolvimento de Aplicação Android para Monitorização de Idosos} at {\sc inesc}. Supervised by \emph{Prof. Angelo Martins}.

\subsection*{scientific initiation supervising}
\noindent
\years{2012 --- 13}Luís Fonseca, \emph{Object-Functional Pattern Variants of Adaptive Object-Models} at {\sc inesc tec}.

\subsection*{other educational~\amper~teaching services}
\noindent
\years{2015}Juror, Tiago Sousa, \emph{``Raise of the Dead''}. Professional Aptitude Evaluation at \emph{Escola Secundária Filipa de Vilhena}.\\
\years{2015}Juror, Bruno Jesus, \emph{``Medionics: Um site dinâmico''}. Professional Aptitude Evaluation at \emph{Escola Secundária Filipa de Vilhena}.\\
\years{2014 --- 15}Organizer, \emph{``My first 3D film''}, 1\st and 2\nd editions. An introduction to visual programming languages using Alice, ages 12–15 @ Universidade Júnior \\
\years{2013 --- 14}Organizer, \emph{``Mummy, daddy, I shrunk my computer!!!''}. 1\st and 2\nd editions. An introduction to the RaspberryPI, electronics and programming, ages 12–15 @ Universidade Júnior \\\
\years{2010}Organizer and Lecturer, \emph{``An Introduction to UML 2.x''}. 10 students @ Criticial Software, S.A.

%\hrule
\section*{Scientific services}

\subsection*{degree-awarding juris in informatics engineering~~(excluding supervising juris) · feup}
\noindent
\years{2015}\emph{Chair, MSc}. Jorge Costa, \emph{A Multi-Objective Approach to Test Suite Reduction}\\
\years{2015}\emph{Chair, MSc}. Paulo Freitas, \emph{Software Repository Mining Analytics to Estimate Software Component Reliability}\\
\years{2014}\emph{Chair, MSc}. Daniela Cardeano, \emph{Data Mining em aplicações de Desenho Racional de Fármacos}\\
\years{2014}\emph{Chair, MSc}. Tiago Mota, \emph{Identificação e Quantificação de Células Oncocíticas em Imagens Microscópicas}\\
\years{2014}\emph{Chair, MSc}. João Ponte, \emph{Deteção de falhas em Servidores de Video on Demand}\\
\years{2014}\emph{Chair, MSc}. Bruno Fernandes, \emph{Improving Software Project Estimates Based on Historical Data }\\
\years{2014}\emph{Chair, MSc}. Linda Padilla, \emph{Transformation of Business Process Models: A Case Study}\\
\years{2013}\emph{Juror, PhD}. Fernando Sérgio Barbosa, \emph{Generic Roles: Reducing Code Replication}\\
\years{2013}\emph{Chair, MSc}. Angela Igreja, \emph{AngelMail: an integrated solution for prioritization, visualization and organization of email}\\
\years{2011}\emph{Chair, MSc}. Tiago Carvalho, \emph{A Meta-Language and Framework for Aspect-Oriented Programming}\\
\years{2011}\emph{Chair, MSc}. João Antunes, \emph{Desenvolvimento e integração de editores gráficos de elevado impacto visual}

\subsection*{grant commitees}
\noindent\years{2012}Research grant, \emph{Ambient Assisted Living for All}. 18mo @ {\sc inesc tec} \\
\years{2012}Scientific Initiation grant, \emph{Object-Functional Pattern Variants of AOMs}. 3mo @ {\sc inesc tec}

\subsection*{program commitees \amper~reviews}
\noindent
\years{2015}\emph{Technical Review}. DeVore \amper~Walsh, \emph{``Reactive Application Development''}, Manning Publications Co.\\
\years{2015}\emph{PC Member}. 6\nth Brazilian Workshop on Agile Methods ({\sc wbma @ agile brazil 2015})\\
\years{2014}\emph{PC Member}. 10\nth Latin-American Conf. on Pattern Languages of Prog. ({\sc sugarloafplop} 2014) \\
\years{2014}\emph{PC Member}. Software Engineering and Science track at the 6\nth Symposium on Informatics. ({\sc soft-pt @ inforum 2014}) \\
\years{2014}\emph{PC Member}. 9\nth Int. Conf. on the Quality of Information and Communications Tech. ({\sc quatic} 2014)\\
\years{2014}\emph{PC Member}. Workshop of Agile Software Development Techniques at the 14\nth International Conference on Computational Science and its Applications ({\sc iccsa} 2014)\\
\years{2013}\emph{PC Member}. 5\nth Int. Workshop on Flexible Modeling Tools {(\sc flexitools @ splash 2013)}\\
\years{2013}\emph{PC Member}. MiniPLoP Brasil {\sc ime/usp}\\
\years{2013}\emph{Reviewer}. 18\nth European Conference on Pattern Languages of Programs {(\sc europlop 2013)}\\
\years{2013}\emph{Technical Review}. Alvin Alexander, \emph{``Scala Cookbook''}, O'Reilly Media, Inc.\\
\years{2013}\emph{PC Member}. Wavefront Experience at the 4\nth annual {\sc splash} conference\\
\years{2012}\emph{Reviewer}. {\sc ieee} Transactions on Software Engineering {(\sc tsm)}\\
\years{2012}\emph{PC Member}. 8\nth Int. Conf. on the Quality of Information and Communications Tech. ({\sc quatic} 2012)\\
\years{2012}\emph{Reviewer}. 19\nth Conference on Pattern Languages of Programs {(\sc plop 2012)}\\
\years{2012}\emph{PC Member}. Doctoral Symposium in Informatics Engineering ({\sc dsie} 2012)\\
\years{2011}\emph{PC Member}. 2\nd International Summer School on Domain Specific Modeling\\
\years{2011}\emph{PC Member}. 18\nth Conference on Pattern Languages of Programs {(\sc plop 2011)}\\
\years{2011}\emph{PC Member}. Model Driven Engineering {\sc (ecm/mde, inforum 2011)}\\
\years{2011}\emph{PC Member}. MiniPLoP Brasil {\sc ime/usp}\\
\years{2011}\emph{Reviewer}. 16\nth European Conference on Pattern Languages of Programs {(\sc europlop 2011)}\\
\years{2011}\emph{Reviewer}. {\sc ieee} Software\\
\years{2010}\emph{PC Member}. 6\nth Int. Conf. on the Quality of Information and Communications Tech. ({\sc quatic} 2010)\\
\years{2010}\emph{Reviewer}. 17\nth Conference on Pattern Languages of Programs {(\sc plop 2010)}\\
\years{2010}\emph{Reviewer}. 15\nth European Conference on Pattern Languages of Programs {(\sc europlop 2010)}

\subsection*{conference organization}
\noindent\years{2015}\emph{Scientific Committee}. Inter-University Programming Marathon ({\sc miup} 2015), Porto@{\sc fcup}, Portugal.\\
\years{2014}\emph{Chair}. Inter-University Programming Marathon ({\sc miup} 2014), Porto@{\sc feup}, Portugal.\\
\years{2012}\emph{Co-Chair}. 3\rd Agile Portugal\\
\years{2011}\emph{Co-Chair}. 2\nd Agile Portugal\\
\years{2010}\emph{Co-Chair}. 1\st Agile Portugal\\
\years{2010}\emph{Co-Organizing Chair}. Quality Evolution in {\sc ict} Track of the 7\nth International Conference on the Quality of Information and Communications Technology {(\sc quatic 2010)}\\
\years{2008}\emph{Student Volunteer}. 23\rd Annual {\sc acm sigplan} International Conference on Object-Oriented Programming, Systems, Languages, and Applications {(\sc oopsla 2008)}\\
\years{2008}\emph{Student Volunteer}. 15\nth Conference on Pattern Languages of Programs {(\sc plop 2008)}

\subsection*{affiliations}
\noindent\years{\current~2009 ---}\emph{Member}. The International Hillside Group is an educational non-profit organization, that sponsors and helps running various conferences (PlopConference, EuroPlop, ChiliPlop, KoalaPlop, Mensore PLoP, SugarloafPLoP, and UP97) and has been responsible for getting the \emph{Pattern Languages Of Program Design} series of books put together and published \\
\years{\current~2008 ---}\emph{Member}. Software Engineering Group at Faculty of Engineering, University of Porto

\subsection*{other scientific~\amper~management activities}
\noindent
\years{\current~2013 ---}\emph{{\sc acm icpc} Trainer}. International collegiate programming contest trainer of {\sc feup} teams.\\
\years{\current~2011 ---}\emph{Head of Laboratory}. Masters in Informatics and Computation Engineering Student's Lab {\sc feup}.\\
\years{2012 --- 13} Self-assessment Committee for the {\sc a3es} accreditation of the {\sc mieic} course at {\sc feup}. \\
\years{2011 --- 12} Self-assessment Committee for the {\sc eur-ace} accreditation of the {\sc mieic} course at {\sc feup}.

\section*{Main Projects}

\subsection*{research projects}
\noindent
\years{2011 --- 14}{\sc aal4all: ambient assisted living for all, inesc porto}, €8.18m \emph{cf.} {\sc qren 13852} \\
Senior Researcher. A mobilizing project for an industrial ecosystem of products and services in the scope of Ambient Assisted Living {\sc (aal)}, focused on the definition of specific standards. \\
\years{2011 --- 13}{\sc caalyx-mv: complete aal market validation, inesc porto}, €3.91m \emph{cf.} {\sc ict-psp --- 09-3-250577}\\
Senior Researcher. An european project to widely validate an innovative and efficient {\sc ict}-based solution focused on improving the quality of life of the elderly through wearable light devices. \\
\years{2011 --- 12}{\sc ecaalyx: enhanced aal, inesc porto}, €2.7m \emph{cf.} {\sc aal-010000 --- 09-13}\\
Senior Researcher. Three-year project funded by the European Commission under the AAL Joint Programme (Strategic Objectives addressed: ICT-based Solutions for Prevention and Management of Chronic Conditions of Elderly People). The project builds on the strengths of the infrastructure and functionality already developed in the original {\sc caalyx} project (2007/2008).

\subsection*{military research \amper~development}
\noindent
\years{2004 --- 05}{\sc mrs: mip reference system, nc3a} \\
Architecture, design and implementation of the {\sc mip} reference system for {\sc lc2idem} data replication, used by the \acr{nato} \rnd Agency, and to which the alliance implementations have to comply. \\
\years{2003 --- 04}{\sc mipx: framework for development of command and control systems, nc3a} \\
Architecture, design and implementation of an infrastructure to develop {\sc lc2idem}-based Command and Control Systems.

\subsection*{research projects within commercial environment}
\noindent
\years{\current~2012 ---}{\sc ad-forecaster, shiftforward} \emph{cf.} {\sc sifide 0694/2013-e}~\inurl{http://www.adforecaster.com}\\
Chief Technology Officer {(\sc cto)}. A next-generation forecasting engine for online ad campaigns that overcomes key limitations of existing forecast engines, by allowing accurate prediction of future ad impressions traffic levels and campaign inventory availability using unlimited number of targeting variables, including geo, keywords, key-values, cookies, and multiple frequency capping groups at banner, booking, line item or campaign level. Technologies: Scala, Akka, {\sc jvm}, Amazon {\sc ec2}. \\
\years{2012 --- 13}{\sc ad-stress, shiftforward} \emph{cf.} {\sc sifide 0694/2013-e}~\inurl{http://www.adstress.com}\\
Chief Technology Officer {(\sc cto)}. A platform that effeciently simulates complete browser interactions with advertising platforms at a very large scale (over 100m unique users) with multiple interactions with the system. It can also simulate server-to-server connections for testing {\sc rtb} Clients and Servers at very high requests per second (over 50k). Technologies: Scala, Akka, {\sc jvm}, Amazon {\sc ec2}. \\
\years{2007 --- 10}{\sc oghma, paradigmaxis} \emph{cf.} {\sc sfrh/bde/33298/2008} \\
One of the results from my Ph.D. work, it is the current main infrastructure of several production-level Information Systems developed at ParadigmaXis, including Locvs, Zephyr and {\sc gisa}. Technologies: {\sc c\#}, SQLite, {\sc xml}.

\subsection*{commercial projects}
\noindent
\years{\current~2014 ---}{\sc ad-stax, shiftforward}~\inurl{http://www.adstax.com}\\
Chief Technology Officer {(\sc cto)}. Adstax is a scalable, real-time advertising platform-as-a-service, fully customisable and API-accessible, that enpowers a company to integrate 1st and 3rd party technologies within its walled garden. \\
\years{2012}{\sc re-engineering of medical software, im3dical} \\
Project Leader and Senior Consultant for {\sc im3dical} re-engineering of its medical software for Mac OSX and iOS platforms. Technologies: Objective-C, {\sc xml}. \\
\years{2012}{\sc hl7 integration, im3dical} \\
Project Leader and Senior Consultant for {\sc im3dical} integration with {\sc hl7}, developed at {\sc inesc} Porto. Technologies: Microsoft\reg~BizTalk, {\sc xml}, {\sc hl7}, {\sc mllp}. \\
\years{2011 --- 11}{\sc coolbiz: trading system, banco carregosa} \\
Software Quality Engineer for a back-end and front-end financial trading system for \emph{Banco Carregosa}, developed at ParadigmaXis. Technologies: {\sc \cpp}, Apache {\sc qpid}, {\sc amqp}, {\sc qt}, {\sc xml}. \\
\years{2009 --- 10}{\sc smqvu: urban life quality, cmp} \\
Analysis and development of an Information System for \emph{Câmara Municipal do Porto – Gabinete de Estudos e Projectos}. Supports the assessment of urban life quality attributes of the city of Porto, by providing a centralized tool to collect, analyze and synthesize statistical data over hundreds of indexes. Technologies: {\sc c\#}, {\sc wpf}, Oracle database 11g, SQLite, {\sc qt}, {\sc xml}. \\
\years{2007 --- 10}{\sc locvs: architectural and archaeological heritage, cmp} \\
Analysis and development of an Information System for \emph{Câmara Municipal do Porto, Departamento Municipal do Património e Cultura}. It supports the full workflow of collection, process, analysis and storage of textual, graphical and geospatial heritage information of the city of Porto. Technologies: {\sc c\#}, {\sc wpf}, Oracle database 11g, SQLite, {\sc arcgis}, {\sc xml}. \\
\years{2006 --- 07}{\sc geoxis: geo-spatial back-end framework and infrastructure, paradigmaxis} \\
Design and implementation of an infrastructure for displaying and analysing geospatial data. Used by several partners including (but not limited to): Vodafone, Portugal Telecom, Optimus, Clix and {\sc itp}. Technologies: Python, {\sc \cpp}, {\sc c\#}, {\sc html5}, Javascript, PostgreSQL, PostGIS, {\sc xml}. \\
\years{2005 --- 06}{\sc mapadventure, paradigmaxis}~\inurl{http://www.mapadventure.com.pt}\\
Design and implementation of an application and infrastructure for displaying and using military cartography in mobile devices, intented for sports and leisure activities in the wild. The system was developed in cooperation with Infoportugal and IGeOE. Technologies: {\sc \cpp}, {\sc c\#}.

\vspace{1cm}
\begin{adjustwidth}{-1.75cm}{}
\section*{Teaching}
\vspace{-1.1cm}

\newcolumntype{C}[1]{>{\centering\hspace{0pt}}m{#1}}
\hspace{-1.9mm}\begin{tabular}{ p{6.3cm} c c c c c c c c c }
  & \multicolumn{6}{c}{\sc\footnotesize assistant lecturer} & \multicolumn{3}{c}{\sc\footnotesize assistant professor} \\
  \cmidrule(lr){2-7} \cmidrule(lr){8-10}
  & {\footnotesize 07-08} & {\footnotesize 08-09} & {\footnotesize 09-10} & {\footnotesize 10-11} & {\footnotesize 11-12} & {\footnotesize 12-13} & {\footnotesize 13-14} & {\footnotesize 14-15} & {\footnotesize 15-16} \vspace{1mm}\\
  \multicolumn{7}{l}{\color{feup}\sc integrated master in informatics and computing engineering · feup · 5 years\vspace{0.2cm}} \\
  Agile Software Development Methodologies & · & ·   & ·   & ·    & \tp  & \tp  & \tp  & ·   & ·   \\
  Distributed Systems              		     & · & ·   & ·   & ·    & ·    & ·    & ·    & ·   & \tp \\
  Formal Methods in Software Engineering   & · & \tp & \tp & \tp  & ·    & ·    & ·    & \tp & \tp \\
  Software Engineering Seminars 		       & · & ·   & ·   & ·    & \tp  & ·    & ·    & ·   & ·   \\
  Object-Oriented Programming Laboratory   & · & ·   & ·   & \tp  & ·    & ·    & \tp  & ·   & \tp \\
  Operating Systems 					             & · & ·   & \tp & ·    & ·    & ·    & ·    & ·   & \tp \\
  Project Management Laboratory 		       & · & ·   & ·   & \tpp & ·    & ·    & ·    & \tp & ·   \\
  Software Engineering 					           & · & \tp & ·   & ·    & ·    & \tp  & \tp  & \tp & \tp \\
  Software Engineering Laboratory 		     & · & \tp & ·   & ·    & ·    & ·    & ·    & ·   & \tp \\
  \\
  \multicolumn{7}{l}{\color{feup}\sc integrated master in electrical and computers engineering · feup · 5 years\vspace{0.2cm}} \\
  Programming 2 						               & · & ·   & ·   & ·    & \ttp & \ttp & \tpp & ·   & ·   \\
  Formal Methods in Software Engineering   & · & \tp & \tp & \tp  & ·    & ·    & ·    & ·   & ·   \\
  Operating Systems 					             & · & ·   & ·   & ·    & ·    & ·    & \tp  & ·   & ·   \\
  \\
  \multicolumn{7}{l}{\color{feup}\sc master in services engineering and management · feup · 2 years\vspace{0.2cm}} \\
  Requirements Engineering for Services    & · & ·   & ·   & ·    & ·    & ·    & ·    & \tp & ·   \\
  \\
  \multicolumn{7}{l}{\color{feup}\sc licenciate in information science · feup \amper~flup · 3 years\vspace{0.2cm}} \\
  Information Systems Analysis 2           & · & ·   & ·   & ·    & \tpp & \tpp & \tpp & ·   & ·   \\
  \\
  \multicolumn{7}{l}{\color{feup}\sc post-graduation in enterprise application engineering · isep · 1 year\vspace{0.2cm}} \\
  Agile Software Development Methodologies & · & ·   & ·   & \tpp & ·    & ·    & ·    & ·   & ·   \\
  \\
  \multicolumn{7}{l}{\color{feup}\sc licenciate in informatics engineering · istec · 3 years\vspace{0.2cm}}\\
  Database Management Systems              & \ttp & ·& ·   & ·    & ·    & ·    & ·    & ·   & ·   \\
\end{tabular}

\end{adjustwidth}

\end{document}
